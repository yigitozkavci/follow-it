\documentclass{article}
\usepackage[utf8]{inputenc}
\usepackage{graphicx}
\graphicspath{ {images/} }
\usepackage{listings}
\usepackage{color}
\usepackage{textcomp}
\usepackage{xcolor}
\usepackage{hyperref}

\title{CMPE436 Final Project - Follow-it}
\author{
  Yiğit Özkavcı \\
  \texttt{2013400111} \\
  \texttt{yigit.ozkavci@boun.edu.tr}
}
\date{December 2017}

\begin{document}

\definecolor{mygreen}{rgb}{0,0.6,0}
\definecolor{mygray}{rgb}{0.5,0.5,0.5}
\definecolor{lightgray}{rgb}{0.9,0.9,0.9}
\definecolor{mymauve}{rgb}{0.58,0,0.82}
\colorlet{punct}{red!60!black}
\definecolor{background}{HTML}{EEEEEE}
\definecolor{delim}{RGB}{20,105,176}
\colorlet{numb}{magenta!60!black}

\lstset{ %
  backgroundcolor=\color{lightgray},   % choose the background color; you must add \usepackage{color} or \usepackage{xcolor}; should come as last argument
  basicstyle=\footnotesize,        % the size of the fonts that are used for the code
  breakatwhitespace=false,         % sets if automatic breaks should only happen at whitespace
  breaklines=true,                 % sets automatic line breaking
  captionpos=b,                    % sets the caption-position to bottom
  commentstyle=\color{mygreen},    % comment style
  deletekeywords={...},            % if you want to delete keywords from the given language
  escapeinside={\%*}{*)},          % if you want to add LaTeX within your code
  extendedchars=true,              % lets you use non-ASCII characters; for 8-bits encodings only, does not work with UTF-8
  keepspaces=true,                 % keeps spaces in text, useful for keeping indentation of code (possibly needs columns=flexible)
  keywordstyle=\color{blue},       % keyword style
  language=Octave,                 % the language of the code
  morekeywords={*,...},            % if you want to add more keywords to the set
  rulecolor=\color{black},         % if not set, the frame-color may be changed on line-breaks within not-black text (e.g. comments (green here))
  showspaces=false,                % show spaces everywhere adding particular underscores; it overrides 'showstringspaces'
  showstringspaces=false,          % underline spaces within strings only
  showtabs=false,                  % show tabs within strings adding particular underscores
  stepnumber=2,                    % the step between two line-numbers. If it's 1, each line will be numbered
  stringstyle=\color{mymauve},     % string literal style
  tabsize=2,	                     % sets default tabsize to 2 spaces
  title=\lstname,                  % show the filename of files included with \lstinputlisting; also try caption instead of title
    numbers=none,
    showstringspaces=false,
    breaklines=true,
    frame=lines,
    backgroundcolor=\color{background},
}

\lstdefinelanguage{json}{
    numbers=none,
    showstringspaces=false,
    breaklines=true,
    frame=lines,
    backgroundcolor=\color{background},
    literate=
     *{0}{{{\color{numb}0}}}{1}
      {1}{{{\color{numb}1}}}{1}
      {2}{{{\color{numb}2}}}{1}
      {3}{{{\color{numb}3}}}{1}
      {4}{{{\color{numb}4}}}{1}
      {5}{{{\color{numb}5}}}{1}
      {6}{{{\color{numb}6}}}{1}
      {7}{{{\color{numb}7}}}{1}
      {8}{{{\color{numb}8}}}{1}
      {9}{{{\color{numb}9}}}{1}
      {:}{{{\color{punct}{:}}}}{1}
      {,}{{{\color{punct}{,}}}}{1}
      {\{}{{{\color{delim}{\{}}}}{1}
      {\}}{{{\color{delim}{\}}}}}{1}
}

\maketitle

\tableofcontents

\newpage

\section{Abstract}
Follow-it is an Android application that lets you get information on what someone is posting on their Twitter account. Along with that, Follow-it lets you get continuous updates whenever someone tweets something, by making use of sockets.

In the following sections, we will describe the behavior of both server and client, and how we encode the data over the sockets in a useful manner.

\section{Introduction}
\par Motivation of creating Follow-it is to be updated whenever someone updates their feed on the Twitter. With the Twitter UI being distracting for this simple achivement, this Android application achieves the task well enough.

\par The application has a simple interface and user flow: a screen with a text field welcomes the user, user enters his/her name and registers. At this time, application sends the request to server via socket, in FIMessage(see \ref{messages}) format.

\par After registration is accepted by the server, user is ready to add new subscriptions. To do this, user clicks to ``Add Subscription'' button and fills the ``username'' information in there. After user's attempt on creating a subscription is completed, it's the time an approval/rejection message comes from the server. At this point, server decides whether it's a valid Twitter username (server makes this test via twitter4j\cite{twitter4j} library) and based on this decision, tells the client whether its subscription creation request is to be fulfilled or not.

\par After subscription creation attempt is completed, user will get an alert box denoting whether the attempt has been successful or not. If successful, a new subscription record is added to both server and client. In server, this subscription info is kept in-memory, and without any databases. In client, though it's different. We use Android's Room\cite{room} persistence library to make sure we don't lose and data while transitioning between activities \& opening and closing the application.

\par In dashboard page, we show user all of his/her subscriptions which s/he can click to find all updates about \textbf{that} subscription. In single subscription page, we present all of the tweet updates that has been received by this client until now.

\par Due to way server handles updates, we have the initial data paged with 20 tweets maximum, and incrementally watch for the updates for every 5 seconds (see section for rate limiting: \ref{rate_limiting}). And server does some optimizations (see \ref{optimizations}) regarding the fetching for updates, because of the mentioned rate limiting.

\newpage

\section{Approach}
We will divide discussing our approach to two different categories: server and client. This is because even though both architectures has many similarities (like message abstraction and connection interface), they also differ in many other areas (like data storage, event handling and type \& quantity of threads).

\subsection{Server}
As entry point in the server, we instansiate Twitter library, ServerSocket and Twitter Channel (see \ref{channels}). All server does after this is to wait for client connections and create a new client thread that communicates with that client, personally.

Creating a client thread is simple from server's side:
\begin{itemize}
  \item Accept a connection from ServerSocket
  \item Prepare input and output streams
  \item Start the client thread with these gathered parameters (see \ref{client_thread})
\end{itemize}

\subsubsection{Client Thread}
\label{client_thread}
\par A client thread is a thread which is responsible of all the communication going on between server and THAT client. Since we don't need a client-to-client communication, client threads are completely independent of each other.

\par Client thread is where the magic happens. Client thread listens the inputstream Server gave it while creating the socket connection with the client. Whenever something it written into inputstream, client picks it and processes it. We have a function with signature:
\begin{lstlisting}[firstline=1, language=Java]
Optional<Command> processInput(String inputLine)
\end{lstlisting}
\par As the obvious signature suggests, this function takes the input and \textbf{maybe} generates a command (see \ref{commands}) out of it. If client thread is indeed able to extract the command, it calls $perform()$ on command instance right away. 

\par Before discussing commands, though, we should talk about handling messages by the server. After that, we will discuss implementation of those messages, namingly commands.

\newpage
\subsubsection{Messages}
\label{messages}
Server has defined two types of messages, with two distinct enum types for their tags: $Server.Tag$ and $Client.Tag$.

\par A tag describes what that message wants to do. Every message in Follow-it must carry a valid tag along with it, and other fields are not obligatory, even though we have a convention of sending messages along with a data key. We will provide examples for each message type, so our convention will be obvious.\\

\textbf{Client to Server Messages:}
\begin{itemize}
  \item \textbf{REGISTER}: Sent when client wants to register. A data with key ``username'' should be sent along with the message, indicating the username that client wants to register with.
    \par Server then determines whether registration is successful or not (it always is, as we don't have a predicate for registration-checking, unlike subscription. See REGISTER\_ACCEPT for more.). An example:
\begin{lstlisting}[language=json]
{
  "tag": "REGISTER",
  "data": {
    "username": "Yigit"
  }
}
\end{lstlisting}
  \item \textbf{SUBSCRIBE}: Sent when client wants to subscribe to a channel, for a user. Server then determines whether user is eligible to subscribe to the provided username for the provided channel. An example:
\begin{lstlisting}[language=json]
{
  "tag": "SUBSCRIBE",
  "data": {
    "channel": "TWITTER",
    "username": "LondraGazete"
  }
}
\end{lstlisting}
\end{itemize}

\textbf{Server to Client Messages:}
\begin{itemize}
  \item \textbf{REGISTER\_ACCEPT}: Sent by the server if server decides that client can be registered. This is always the case, since we don't check eligibility of the registration. We could have rules like username being unique though.
  \item \textbf{SUBSCRIPTION\_ACCEPT}: Sent by the server if server decides that client's subscription attempt is valid and accepted. Server decides whether a subscription attempt is valid by checking the Twitter username via Twitter4j API library. If user doesn't exist, a SUBSCRIPTION\_REJECT message is sent instead.
  \item \textbf{SUBSCRIPTION\_REJECT}: Sent by the server if server decides that client's subscription attempt is not valid. This case only happens if the provided username is wrong or a Twitter user with that username doesn't exist. This decision requires us to make a HTTP request to Twitter API, via Twitter4j library.
  \item \textbf{ERROR}: As we stated earlier, we have a pre-defined protocol message format, which includes having a valid json structure and having a valid key ``tag'' and a valid TAG value associated with it, which should be either ``REGISTER'' or ``SUBSCRIBE''. If client fails to send message with this format, server catches this error and immediately notifies the client about its mistake.
    \par Message format being invalid is not the only case for this ERROR message. In specific points such as subscription, server can throw SubscriptionException or ProtocolException. All of these exceptions are always caught and turnt into a valid ``ERROR'' message to notify client properly. Server also logs all of these exception occurences, just in case of debugging.
  \item \textbf{TWEETS}: This message denotes that server wants to send client tweets data, associated with an existing username. When client sees the message, it can be sure that there are tweet updates for the user with username associated with this message. Server doesn't send any empty updates to client if there are no tweet updates. We will discuss scheduling and pollig for tweet messages, as well as the rate limiting optimization in the following sections.
\end{itemize}

\subsubsection{Commands}
\label{commands}
Command is an abstraction over what to be done with a valid client message. If you want to perform a command, it takes action based on the message data it has, which consists of a tag and a hashmap with keys being Strings and values being Objects.

\par For now, only commands server should handle are the messages with tags ``REGISTER'' and ``SUBSCRIBE''. Their behavior highly involves our task management in server, so we are discussing them in depth in the following sections.

\newpage

\subsubsection{Channels}
\label{channels}
Channels are essentially managers for the particular social media (only Twitter for our case). I mentioned that unlike client, we keep our data in-memory in server-side, and TwitterChannel is the object that keeps all the subscription data in its instance variable ``subscriptions'', which is just a mapping from users to list of clients.

\par This hashmap's structure may seem a bit strange at first, but it has a good reason behind it. We will discuss on it at section \ref{rate_limiting}, since it involves the mindset evolved for rate limiting optimization.

\par TwitterChannel is instansiated only once through all the server application. This is because we need only one manager to provide support for us for one social media. After that, the only thing one could do with this instance is to call $subscribe()$ on it. This is the major and only api this object exposes.

\par Once one invokes subscribe on twitter channel, first we find the Twitter user with that username. Then there are two options:
\begin{itemize}
  \item If a subscription for that twitter user has been made before, we add that client to that user's watchers list
  \item Else, we create a new entry on $subscriptions$ hashmap, and spawn a new TwitterAgent for that user. TwitterAgent will be discussed just below.
\end{itemize}

\par In order to watch updates on a certain user, we use Java's Timer class, along with TimerTask for scheduling tasks. The spawned TwitterAgents are subclasses of TimerTasks, and this inheritance allows us to invoke $run()$ method on each TwitterAgent every 5 seconds (we define this interval ourselves, and this is subject to change, see \ref{optimizations}). What this TwitterAgent does it that it makes a request to twitter via Twitter4j\cite{twitter4j} with the username it's built with, and gets the tweets associated with that user. Since we already performed the Twitter user check, from then on, we started using Twitter.User class to denote a real Twitter user, instead of a String denoting an unsafe Twitter username.

\par After we have the tweets, \textbf{we mark this point at tweet feed} for further fetchings not to duplicate tweet fetching. If we have $>0$ amount of tweets in hand, we are ready to notify the watchers. Remember that we had two parameters for instansiating a TwitterAgent: a $Twitter.User$ and $List<Client>$. We call $notifyUpdate()$ on all of the agents with the tweets we fetched, and hence, we pushed the update to all clients.

\newpage

\subsubsection{Rate Limiting}
\label{rate_limiting}
One of the biggest concerns while making an application that consumes a 3rd party api is rate limiting. And Twitter API, which we consume has rate limiting individual to each of its endpoints. The endpoint we make use of most is the one allows us to fetch tweets, and it has a limit of 900 requests per 15 minutes.

\par Given that we make request to Twitter API for fetching tweets per 5 seconds for each individual Twitter user, we can only have 5 concurrent Twitter users being watched. This is a hard constraint in the sense that we have no chance of changing it, since it's a third party API. But we can try to do some optimizations on the subject, which we discuss in section \ref{optimizations}.

\subsubsection{Optimizations}
\label{optimizations}
\par Because of the hard constraint Twitter API dictates, we need find ways of allowing more clients getting concurrent updates. Hence, we did some optimizations regarding amount of requests being sent to client.

\par We build internal data structure of Twitter Channel so that is focuses on a single user. This way, even though there are multiple clients watching for a user, we only fetch the tweet data once every 5 seconds for that user, and broadcast this to all clients. This method has a drawback though: we are unable to synchronize newcoming clients with the latest tweets, since we are only allowed to broadcast new updates for the user. We can achieve this behavior \textbf{partly} by caching the tweets and then sending them as bulk to newcoming clients, but tweets in that data can be deleted or updated by mentions, so there is no guaranteed way to achieve best of the both worlds.

\subsection{Client}
\label{client}
Client architecture is different from the one we have in server, because of the nature of distributed applications, clients are the ones who needs to be proactive and tell the server what they want. In Android applications, this is done with buttons and inputs used by users that cause server to take some action.

\par Before discussing the actual behavior of the Android application, we should talk a little bit about async tasks. 

\newpage 

\subsubsection{Async Programming}
\label{async_programming}

\par Every IO in this Android application is done asynchronously. This is considered the best practice because making a thread wait until some action is taken means blocking the whole activity, and also means that we are blocking the user. Every action taken, including registering and subscribing has a common interface (we are not talking about the actual Java interface) that causes them to pass a \textbf{listener} along with the action being taken. This is called the callback paradigm, which Javascript makes heavy use of.

\par This listener should obey to the generic type provided when passing it to the task. This generic type dictates the return value of the listener, which constraints to the actual task that returns the value. Hence because of this constraint, we can say we want to get tweet data, and the signature of our data is $List<Tweet>$. This ensures the integrity of our program, which is hard to maintain when programming with callbacks / performing async programming.

\par We are using these listeners in 3 places throughout the application: \\

\par \textbf{Registration}: It being called is enough as evidence that registration is completed, no need for an actual result
\begin{lstlisting}[firstline=1, language=Java]
TaskListener<Void>
\end{lstlisting}

\par \textbf{Subscription}: True if subscription succeeded, false otherwise
\begin{lstlisting}[firstline=1, language=Java]
TaskListener<Boolean>
\end{lstlisting}

\par \textbf{Receiving Data}: This may seem a little bit odd, but the triple result actually explains much. It stands for: there is data of tweets coming for \textbf{Channel} channel for \textbf{String} user.
\begin{lstlisting}[firstline=1, language=Java]
TaskListener<Triple<Channel, String, ArrayList<String>>
\end{lstlisting}

\newpage

\subsubsection{Application Behavior}
\label{application_behavior}
\par When Android is booted, we don't open the socket connection right away, because it would be meaningless since user hasn't done anything yet, and there is nothing for server to consider a client as. User sees a welcome page with just an input box and a button, which is used to register user, and hence client to server. User enters his/her username, then we send server a REGISTER message. Then this activity does absolutely nothing, until a ``REGISTER\_ACCEPT'' message comes from the server. When the message arrives, the callback we provided runs $startActivity()$ and we go to dashboard.

\par In the dashboard, we have a list of subscriptions provided. Dashboard activity is responsible of calling the database task ``GetSubscriptions'' and asynchronously fill the subscription data with existing persistent storage. The reason behind deciding on using a persistent storage is that the data we pass with the activities turnt out to be too complex for serializing \& deserializing and passing as intent extra. Because of this, we decided to use Android's Room library for persistent storage, and we've written our tasks, entities and relationships for our models.

\par Database tasks are defined in $tasks$ package, and in order for one to use them, one should provide necessary parameters to the constructor and provide a callback, since this is a asynchronous task. This is indeed when DashboardActivity does on its $onStart()$ call.

\par Android application has other behaviors as well, but they will be demonstrated in presentation.

\section{Experimental Methodology}
While working with a 3rd party API, it's always hard to write automated tests, since you need to mock all the behavior of the API, as it's not in your source code. Luckily, we make a very little use of Twitter API, and that's something we can experiment on easily.

\par We created a new Twitter user for the sake of testing this API, and our server's incremental data pushes. We are the owner of the account \url{https://twitter.com/fi\_debug} and tested the tweet data pushing feature on-hand. Here is the experiment steps:

\begin{enumerate}
  \item Open the application
  \item Write a arbitrary username and click to ``Register'' button
  \item See the empty page, and click to ``Add Subscription'' button
  \item Type the name ``wrong\_user\_name\_203120582'', click ``Add Subscription''
  \item Expect a failure popup message
  \item Repeat steps from 3 to 4 with username ``YigitOzkavci''
  \item Expect a successful popup message
  \item Repeat steps from 3 to 4 with username ``fi\_debug''
  \item Expect a successful popup message
  \item Click on subscription ``YigitOzkavci'' and expect to see latest tweets of the account @YigitOzkavci and go back in Android navigation
  \item Click on subscription ``fi\_debug'' and expect to see latest tweets of the account @fi\_debug and go back in Android navigation
  \item Post a tweet as user @fi\_debug from Twitter UI
  \item Click on subscription ``fi\_debug'' and expect to see newly-posted tweet
\end{enumerate}
\section{Experimental Results}
Experimental results were expected, but it also notified us about the constraint on api rate limiting \& not being able to deliver old tweets to new subscribers (see \ref{optimizations} as a reminder).
\section{Appendix}

\lstinputlisting[language=Java]{../server/src/main/java/followit/Server.java}
\lstinputlisting[language=Java]{../server/src/main/java/followit/Client.java}
\lstinputlisting[language=Java]{../server/src/main/java/followit/channel/Channel.java}
\lstinputlisting[language=Java]{../server/src/main/java/followit/channel/ChannelBuilder.java}
\lstinputlisting[language=Java]{../server/src/main/java/followit/channel/TwitterChannel.java}
\lstinputlisting[language=Java]{../server/src/main/java/followit/command/Command.java}
\lstinputlisting[language=Java]{../server/src/main/java/followit/Utils/Tuple.java}
\lstinputlisting[language=Java]{../server/src/main/java/errors/SubscriptionException.java}

\newpage

\begin{thebibliography}{9}
\bibitem{twitter4j}
  Twitter4j: A Java library for the Twitter API \url{http://twitter4j.org/en/index.html}
\bibitem{room}
  Room: Persistence Library for Android \url{https://developer.android.com/topic/libraries/architecture/room.html}

\end{thebibliography}

\end{document}
